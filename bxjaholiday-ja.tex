% +++
% latex = "uplatex"
% extra_clean_files = [
%   "%B.dvi", "%B.synctex.gz", "%B.hd", "%B.glo"
% ]
% +++
\documentclass[uplatex,dvipdfmx]{wtpl3doc}

% packages
\usepackage[verb]{bxghost}
\usepackage{bxjaholiday}
\usepackage{caption}

% logos
\usepackage{bxtexlogo}
\bxtexlogoimport{*}

% syntax
\newenvironment{code}{\quote\small}{\endquote}

% cross-refs
\usepackage{wtref}
\newref{tab,sec}

% macros
\newcommand{\PkgName}{\pkg{bxjaholiday}}
\newcommand{\Ie}{i.\,e.}
\newcommand{\Eg}{e.\,g.}

\ExplSyntaxOn
\cs_set_eq:NN \_ \c_underscore_str
\cs_new:Npn \__bxjh_doc_holiday_line:nn #1#2
  {
    % first column: holiday
    \tl_use:c { g_bxjh_#2_tl } &
    % second column: variable
    \cs { g_bxjh_#2_tl } \\
  }
\cs_new:Npn \__bxjh_doc_week_of_day_line:nn #1#2
  {
    % first column: week of day
    \tl_use:c { g_bxjh_#2_tl } &
    % second column: int variable
    \cs { c_bxjh_ \tl_lower_case:n{ #1 } _int } &
    % third column: int value
    \int_use:c { c_bxjh_ \tl_lower_case:n{ #1 } _int } &
    % fourth column: tl variable
    \cs { g_bxjh_#2_tl } &
    % fifth column: tl value
    \tl_use:c { g_bxjh_#2_tl } \\
  }
\cs_new_eq:NN \HolidayLine \__bxjh_doc_holiday_line:nn
\cs_new_eq:NN \WeekofdayLine \__bxjh_doc_week_of_day_line:nn
\ExplSyntaxOff

% metadata
\GetFileInfo{bxjaholiday.sty}
\title{{\PkgName}パッケージ}
\author{朝倉卓人 (wtsnjp)}
\date{\fileversion \quad [\filedate]}

\begin{document}

\maketitle

\begin{abstract}
このパッケージは日付から日本の祝日名を得るためのコマンドと,その副産物である
日付から曜日を取得するコマンドを提供する.expl3向けには,これらのコマンドと
同等の機能を有する関数と,追加の低級関数も用意されている.
\end{abstract}

%\tableofcontents

\begin{documentation}

%\section{Introduction}

\section{動作要件}

BXプレフィックス\footnote{BXシリーズは主として八登祟之 (a.k.a.~ZR) 氏によって
開発されている\LaTeX パッケージ群である.``BX'' は ``\underline{b}abel
e\underline{x}tensions'' を意味し,BXシリーズに含まれるパッケージは通常,日本語
に特化したもの(\pTeX 系列)に限らず多様な\TeX 処理系をサポートしている.}の示す
通り,本パッケージは様々な\TeX 処理系のサポートを意図しており,expl3がサポート
するすべての\TeX 処理系をサポートしている(つまり\eTeX 拡張は必要である).
具体的には以下のシステムがサポートされている:
%
\begin{itemize}
\item \TeX フォーマット:\LaTeXe
\item \TeX 処理系:{\pdfTeX}, {\XeTeX}, {\LuaTeX},\footnote{\pTeX 系列以外の
処理系で和文文字を表示するには,別途フォントやその他の設定を行う必要がある.}
{\pTeX}, {\upTeX}
\end{itemize}

\section{パッケージ読み込み}

本パッケージの読み込みは\LaTeXe の標準的な方法で行う.オプションは存在しない.
%
\begin{code}
|\usepackage{bxjaholiday}|
\end{code}

\section{{\LaTeXe}インターフェース}

\begin{function}[EXP]{\jaholidayname}
\begin{syntax}
|\jaholidayname|\marg{year}\marg{month}\marg{day}
\end{syntax}
%
与えられた日付が日本の祝日に該当する場合,その祝日の名前に展開される.日付が
祝日でない場合は展開後には何も残らない(空のトークンに展開される).
表\tabref{holidays}に展開結果となり得る祝日名の一覧を示す.

引数の\meta{year}, \meta{month}, \meta{day}には明示的な数値またはカウンタ
(例えば|\year|, |\month|, |\day|)を与えることができる.正確には,任意の
\meta{integer expression}が許される.
\end{function}

\begin{function}[EXP]{\jadayofweek}
\begin{syntax}
|\jadayofweek|\marg{year}\marg{month}\marg{day}
\end{syntax}
%
日付を曜日名(月・火・水・木・金・土・日のいずれか)に変換する.
引数の仕様は|\jaholidayname|と同一である.
\end{function}

\begin{function}[EXP]{\IfJaHolidayTF,\IfJaHolidayT,\IfJaHolidayF}
\begin{syntax}
|\IfJaHolidayTF|\marg{year}\marg{month}\marg{day}\marg{true code}\marg{false code}
|\IfJaHolidayT|\marg{year}\marg{month}\marg{day}\marg{true code}
|\IfJaHolidayF|\marg{year}\marg{month}\marg{day}\marg{false code}
\end{syntax}
%
祝日か否かの判定を行う.なお,この条件分岐では振替休日も祝日として扱われる.
\end{function}

\section{expl3インターフェース}

本パッケージが提供するすべてのインターフェースは\texttt{bxjh}モジュールに属する.

\subsection{関数}

\begin{function}[EXP]{\bxjh_holiday_name:nnn}
\begin{syntax}
|\bxjh_holiday_name:nnn| \marg{year} \marg{month} \marg{day}
\end{syntax}
%
|\jaholidayname|のexpl3関数版である.日付を祝日名に変換する.
\end{function}

\begin{function}[EXP]{\bxjh_day_of_week_name:nnn,\bxjh_day_of_week:nnn}
\begin{syntax}
|\bxjh_day_of_week_name:nnn| \marg{year} \marg{month} \marg{day}
|\bxjh_day_of_week:nnn| \marg{year} \marg{month} \marg{day}
\end{syntax}
%
|\bxjh_day_of_week_name:nnn|は|\jadayofweek|のexpl3関数版で,日付を曜日名に変換
する.曜日の情報をexpl3内で利用したい場合(例えば条件分岐したい場合)には
|\bxjh_day_of_week:nnn|を用いるのがより適切である.この関数は内部的な
整数値 (\texttt{int}) を返す.この値は本パッケージが提供する定数と比較することが
できる(\secref{variables}節参照).
\end{function}

\begin{function}[EXP,TF]{\bxjh_if_holiday:nnn}
\begin{syntax}
|\bxjh_if_holiday:nnnTF| \marg{year} \marg{month} \marg{day} \marg{true code} \marg{false code}
\end{syntax}
%
|\IfJaHoliday(TF)|のexpl3関数版である.
\end{function}

\begin{function}[EXP]{\bxjh_apply_prev_day:Nnnn,\bxjh_apply_next_day:Nnnn}
\begin{syntax}
|\bxjh_apply_prev_day:Nnnn| \meta{function} \marg{year} \marg{month} \marg{day}
|\bxjh_apply_next_day:Nnnn| \meta{function} \marg{year} \marg{month} \marg{day}
\end{syntax}
%
これらの関数は与えられた日付の前日または翌日の日付に\meta{function}を適用する.
\meta{function}は3つの引数を,年月日の順にとる関数である必要がある.例えば
%
\begin{code}
|\bxjh_apply_next_day:Nnnn \bxjh_holiday_name:nnn { 2019 } { 12 } { 31 }|
\end{code}
%
は以下と同じ結果となる:
%
\begin{code}
|\bxjh_holiday_name:nnn { 2020 } { 1 } { 1 }|
\end{code}
\end{function}

\subsection{変数と定数}\seclabel{variables}

\paragraph{祝日名}

すべてグローバルな\texttt{tl}型変数として提供される(表\tabref{holidays}).
%
\begin{table}[b]
\centering\small
\caption{日本の祝日}
\tablabel{holidays}
\begin{tabular}{ll}
\hline
祝日名 & 変数 \\ \hline
\HolidayLine{New Year's Day}{ganjitsu}
\HolidayLine{Coming of Age Day}{seijin}
\HolidayLine{National Foundation Day}{kenkoku}
\HolidayLine{The Emperor's Birthday}{tennou}
\HolidayLine{Vernal Equinox Day}{shunbun}
\HolidayLine{Showa Day}{showa}
\HolidayLine{Greenery Day}{midori}
\HolidayLine{Constitution Memorial Day}{kenpou}
\HolidayLine{National Holiday}{kokumin}
\HolidayLine{Children's Day}{kodomo}
\HolidayLine{(substitute holiday)}{furikae}
\HolidayLine{Marine Day}{umi}
\HolidayLine{Mountain Day}{yama}
\HolidayLine{Autumnal Equinox Day}{shubun}
\HolidayLine{Respect for the Aged Day}{keirou}
\HolidayLine{Sports Day}{sports}
\HolidayLine{Health and Sports Day}{taiiku}
\HolidayLine{Culture Day}{bunka}
\HolidayLine{Labor Thanksgiving Day}{kinrou}
\HolidayLine{National Mourning of Showa}{showa\_taisou}
\HolidayLine{National Wedding of Akihito}{akihito\_kekkon}
\HolidayLine{National Wedding of Naruhito}{naruhito\_kekkon}
\HolidayLine{Core Enthronement Ceremony}{sokuirei}
\HolidayLine{Coronation Day}{sokui}
\hline
\end{tabular}
\end{table}

\paragraph{曜日名}

本パッケージは曜日を表す際,内部的には整数値を用いている.これらの内部整数値に
対応する\texttt{int}型の定数が定義されている.また各曜日名を格納するグローバル
\texttt{tl}変数も提供されている(表\tabref{dayofweek}).
%
\begin{table}[b]
\centering\small
\caption{曜日}
\tablabel{dayofweek}
\begin{tabular}{c|lr|ll}
\hline
曜日 & 定数 (\texttt{int}) & & 変数 (\texttt{tl}) & \\ \hline
\WeekofdayLine{Monday}{getsu}
\WeekofdayLine{Tuesday}{ka}
\WeekofdayLine{Wednesday}{sui}
\WeekofdayLine{Thursday}{moku}
\WeekofdayLine{Friday}{kin}
\WeekofdayLine{Saturday}{do}
\WeekofdayLine{Sunday}{nichi}
\hline
\end{tabular}
\end{table}

\end{documentation}

%\PrintChanges

\end{document}
% vim: spell:
