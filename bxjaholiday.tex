%#!xelatex
\RequirePackage{fontspec}
\documentclass{l3doc}

% packages
\usepackage{bxjaholiday}
\usepackage{caption}

% fonts
\newfontfamily{\fIpaex}{IPAexMincho}[Scale=0.95]
\newcommand*{\Ja}[1]{{\fIpaex#1}}

% logos
\usepackage{bxtexlogo}
\bxtexlogoimport{*}

% syntax
\newenvironment{code}{\quote\small}{\endquote}

% cross-refs
\usepackage{wtref}
\newref{tab,sec}

% macros
\newcommand{\PkgName}{\pkg{bxjaholiday}}
\newcommand{\Ie}{i.\,e.}
\newcommand{\Eg}{e.\,g.}

\ExplSyntaxOn
\cs_set_eq:NN \_ \c_underscore_str
\cs_new:Npn \__bxjh_doc_holiday_line:nn #1#2
  {
    % first column: holiday
    #1 &
    % second column: variable
    \cs { g_bxjh_#2_tl } &
    % third column: Name in Japanese
    \Ja { \tl_use:c { g_bxjh_#2_tl } } \\
  }
\cs_new:Npn \__bxjh_doc_week_of_day_line:nn #1#2
  {
    % first column: week of day
    #1 &
    % second column: int variable
    \cs { c_bxjh_ \tl_lower_case:n{ #1 } _int } &
    % third column: int value
    \int_use:c { c_bxjh_ \tl_lower_case:n{ #1 } _int } &
    % fourth column: tl variable
    \cs { g_bxjh_#2_tl } &
    % fifth column: tl value
    \Ja { \tl_use:c { g_bxjh_#2_tl } } \\
  }
\cs_new_eq:NN \HolidayLine \__bxjh_doc_holiday_line:nn
\cs_new_eq:NN \WeekofdayLine \__bxjh_doc_week_of_day_line:nn
\ExplSyntaxOff

% metadata
\GetFileInfo{\jobname.sty}
\title{The {\PkgName} package}
\author{Takuto ASAKURA (wtsnjp)}
\date{\fileversion \quad [\filedate]}

\begin{document}

\maketitle

\begin{abstract}
This package provides a command to convert dates to names of Japanese holidays
(\emph{shukujitsu}; \Ja{祝日}). For internal use, I need to implement a
function to judge the day of week (\emph{youbi}; \Ja{曜日}), so a command
converting dates to \emph{youbi} in the same manner is also available as a free
gift. The equivalent functions and further (lower-level) APIs are provided for
expl3.
%All {\LaTeXe} and expl3 interfaces are (first-)fully expandable, so
%those could be used almost anywhere in your documents.
\end{abstract}

%\tableofcontents

\begin{documentation}

%\section{Introduction}

\section{System requirements}

As one of the BX series\footnote{BX series is a collection of {\LaTeX} packages
mainly developed by Takayuki YATO (a.k.a.~ZR.) ``BX'' stands for
``\underline{b}abel e\underline{x}tensions'' and packages in this series
normally support various {\TeX} engines not only Japanese-specific ones
({\pTeX}, {\upTeX}, and so on.)} packages, {\PkgName} supports all {\TeX}
engines which supported by expl3 (\Ie, the {\eTeX} extension is required.)
Specifically, following {\TeX} systems are supported:
%
\begin{itemize}
\item {\TeX} format: {\LaTeXe}.
\item {\TeX} engine: {\pdfTeX}, {\XeTeX}, {\LuaTeX},\footnote{Note that if you
want to print Japanese characters with {\TeX} engines which is not specifically
designed for Japanese, you need to setup proper fonts and other things.}
{\pTeX}, and {\upTeX}.
\end{itemize}

\section{Loading the package}

The package should be loaded in the usual {\LaTeXe} way. No package option is
available.
%
\begin{code}
|\usepackage{bxjaholiday}|
\end{code}

\section{{\LaTeXe} interfaces}

\begin{function}[EXP]{\jaholidayname}
\begin{syntax}
|\jaholidayname|\marg{year}\marg{month}\marg{day}
\end{syntax}
%
This command is expanded to the name of Japanese holiday corresponding to the
specified date, if it is a holiday. For a date which is not a holiday, it will
be expanded to nothing (an empty token.) See Table~\tabref{holidays} for all
possible results.

For \meta{year}, \meta{month}, and \meta{day}, you can explicitly write
numbers, or use counters, \Eg, |\year|, |\month|, and |\day|. To be exact,
those could be any \meta{integer expression}.
\end{function}

\begin{function}[EXP]{\jadayofweek}
\begin{syntax}
|\jadayofweek|\marg{year}\marg{month}\marg{day}
\end{syntax}
%
This command converts from a date to the name of week, \Ie, one of \Ja{月, 火,
水, 木, 金, 土, 日}. You can specify the arguments in exactly the same way as
|\jaholidayname|.
\end{function}

\begin{function}[EXP]{\IfJaHolidayTF,\IfJaHolidayT,\IfJaHolidayF}
\begin{syntax}
|\IfJaHolidayTF|\marg{year}\marg{month}\marg{day}\marg{true code}\marg{false code}
|\IfJaHolidayT|\marg{year}\marg{month}\marg{day}\marg{true code}
|\IfJaHolidayF|\marg{year}\marg{month}\marg{day}\marg{false code}
\end{syntax}
%
The |\IfJaHoliday(TF)| tests are used to check if a date is a Japanese holiday
or not. Note that substitute holidays (\Ja{振替休日}) are also judged as a
holiday in this test.
\end{function}

\section{expl3 interfaces}

All expl3 interfaces provided by {\PkgName} belong to the \texttt{bxjh} module.

\subsection{Functions}

\begin{function}[EXP]{\bxjh_holiday_name:nnn}
\begin{syntax}
|\bxjh_holiday_name:nnn| \marg{year} \marg{month} \marg{day}
\end{syntax}
%
This is expl3 version of |\jaholidayname|. It converts dates into Japanese
holiday names.
\end{function}

\begin{function}[EXP]{\bxjh_day_of_week_name:nnn,\bxjh_day_of_week:nnn}
\begin{syntax}
|\bxjh_day_of_week_name:nnn| \marg{year} \marg{month} \marg{day}
|\bxjh_day_of_week:nnn| \marg{year} \marg{month} \marg{day}
\end{syntax}
%
|\bxjh_day_of_week_name:nnn| is an expl3 version of |\jadayofweek|. It converts
a date into day of week in Japanese. To use that information in expl3, \Eg, for
branching, |\bxjh_day_of_week:nnn| is more suitable. It returns an internal
\texttt{int} value, so you can compare those results with the constants provided
by this package. See Section~\secref{variables}.
\end{function}

\begin{function}[EXP,TF]{\bxjh_if_holiday:nnn}
\begin{syntax}
|\bxjh_if_holiday:nnnTF| \marg{year} \marg{month} \marg{day} \marg{true code} \marg{false code}
\end{syntax}
%
This test is expl3 version of |\IfJaHoliday(TF)|.
\end{function}

\begin{function}[EXP]{\bxjh_apply_prev_day:Nnnn,\bxjh_apply_next_day:Nnnn}
\begin{syntax}
|\bxjh_apply_prev_day:Nnnn| \meta{function} \marg{year} \marg{month} \marg{day}
|\bxjh_apply_next_day:Nnnn| \meta{function} \marg{year} \marg{month} \marg{day}
\end{syntax}
%
These functions get previous/next day of the specified date, and apply it to
the specified \meta{function}. The \meta{function} must take three arguments in
the order. For example,
%
\begin{code}
|\bxjh_apply_next_day:Nnnn \bxjh_holiday_name:nnn { 2019 } { 12 } { 31 }|
\end{code}
%
produces the result of:
%
\begin{code}
|\bxjh_holiday_name:nnn { 2020 } { 1 } { 1 }|
\end{code}
\end{function}

\subsection{Variables and constants}\seclabel{variables}

\paragraph{Names of Japanese holidays}

All of them are provided as global \texttt{tl} variables. See
Table~\tabref{holidays}.
%
\begin{table}[p]
\centering\small
\caption{Japanese holidays}
\tablabel{holidays}
\begin{tabular}{lll}
\hline
Holiday & Variable & Name in Japanese \\ \hline
\HolidayLine{New Year's Day}{ganjitsu}
\HolidayLine{Coming of Age Day}{seijin}
\HolidayLine{National Foundation Day}{kenkoku}
\HolidayLine{The Emperor's Birthday}{tennou}
\HolidayLine{Vernal Equinox Day}{shunbun}
\HolidayLine{Showa Day}{showa}
\HolidayLine{Greenery Day}{midori}
\HolidayLine{Constitution Memorial Day}{kenpou}
\HolidayLine{National Holiday}{kokumin}
\HolidayLine{Children's Day}{kodomo}
\HolidayLine{(substitute holiday)}{furikae}
\HolidayLine{Marine Day}{umi}
\HolidayLine{Mountain Day}{yama}
\HolidayLine{Autumnal Equinox Day}{shunbun}
\HolidayLine{Respect for the Aged Day}{keirou}
\HolidayLine{Sports Day}{sports}
\HolidayLine{Health and Sports Day}{taiiku}
\HolidayLine{Culture Day}{bunka}
\HolidayLine{Labor Thanksgiving Day}{kinrou}
\HolidayLine{National Mourning of Showa}{showa\_taisou}
\HolidayLine{National Wedding of Akihito}{akihito\_kekkon}
\HolidayLine{National Wedding of Naruhito}{naruhito\_kekkon}
\HolidayLine{Core Enthronement Ceremony}{sokuirei}
\HolidayLine{Coronation Day}{sokui}
\hline
\end{tabular}
\end{table}

\paragraph{Day of week}

Internally, {\PkgName} uses integers to represent day of week, and corresponding
\texttt{int} constants are defined. In addition to that, Japanese names of those
are also provided as global \texttt{tl} variables. See
Table~\tabref{dayofweek}.
%
\begin{table}[p]
\centering\small
\caption{Day of week}
\tablabel{dayofweek}
\begin{tabular}{l|lr|ll}
\hline
Day of week & Constant (\texttt{int}) & & Variable (\texttt{tl}) & \\ \hline
\WeekofdayLine{Monday}{getsu}
\WeekofdayLine{Tuesday}{ka}
\WeekofdayLine{Wednesday}{sui}
\WeekofdayLine{Thursday}{moku}
\WeekofdayLine{Friday}{kin}
\WeekofdayLine{Saturday}{do}
\WeekofdayLine{Sunday}{nichi}
\hline
\end{tabular}
\end{table}

\end{documentation}

%\PrintChanges

\end{document}
% vim: spell:
